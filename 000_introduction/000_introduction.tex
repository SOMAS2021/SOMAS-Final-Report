\chapter{Abstract}\label{abstract}

\section{Abstract}

Co-operative survival games refer to a subset of political choice games wherein `players' must work together to overcome disaster, else suffer the consequences through both personal and communal damage. Furthermore, without the ability to monitor a system and in the absence of a centralised authority, it becomes impossible to form enduring self-governing institutions, resulting in the proposed self-organising mechanisms for `solving' such games potentially functioning in harmful ways. This paper utilises a self-organising, multi-agent system to simulate an iterated collective action, co-operative survival game comprising a resource management problem, the `solution' to which involves a system of localised governance. Through a series of experiments where the behaviour of the players in the system is increasingly randomised and the presence of self-organising mechanisms is toggled, we investigate the causes of instability. Ultimately, we conclude that, through the integration of governance, it is possible for such a system to effectively self-organise to minimise the total deaths of the players involved.

\chapter{Introduction}\label{introduction}

Co-operative survival games refer to a subset of political choice games wherein `players' must work together to overcome disaster, else suffer the consequences through both personal and communal damage. Instances of co-operative survival exist in varying media: computer games (\emph{e.g.}, ``Don't Starve,'' ``Rust,'' ``Minecraft''), board games (\emph{e.g.}, Ravine) or sociological constructs (\emph{e.g.}, Reducing viral transmission ) all of which exhibit a necessity for collective action, with such notions of survival supported by extensive literature \cite{su10030652}. Survival games are often identified within the context of ``Common Pool Resource Management'' (CPR) \cite{ostromCollectiveAction} \cite{smartHome} problems, where to facilitate co-operative survival, a pool of shared resources must be maintained to help mitigate disaster. In the general case of such survival games, following a disaster, if 1 player dies, all players die.

This project proposes a twofold bifurcation from conventional CPR problems: firstly, there are no provisions made to the common pool, meaning that we cannot refer to this problem as a linear public goods (LPG) game. Instead, players must maintain a collective resource without offering replenishment from their personal pool. This effectively constructs a scenario where all players make appropriation with insufficient provision, yielding a system of \textit{N free-riders}. 

Secondly, Ostrom notes that common-pool management problems are provably solvable through the introduction of self governing institutions \cite{ostromCollectiveAction}, offering a contradiction to the tragedy of the commons and zero contribution thesis \cite{zeroCon}. This project instead offers a scenario without centralised governance and an inability to monitor other players, effectively preventing this research from complying with Ostrom’s ``design principles for enduring self-governing institutions.''

The absence of a centralised authority, coupled with an inability to form self-governing institutions proposes a seemingly unsolvable problem: a common pool of resources must be maintained across a network of players, each acting rationally to maximise local utility \cite{oberUtil}, without the prospect of forming enforceable rules. This necessitates the presence of a form of self-organisation sufficiently powerful as to overcome the challenges set by the nature of this problem, whilst accepting the risk that such a self-organising mechanism may act perniciously under these circumstances \cite{dobSteg}.

Apropos of Artificial Societies, to combat the difficulties poised by such an unconventional CPR game we introduce a notion of self-organisation through governance in the form of treaties. Serving both ``as a juristic act and as a rule'' \cite{reuter1995introduction}, treaties act as a form of institutionalised power between players, with society coming to a mutual agreement upon the importance of such a legal device. Overall, it is the topic of this research to evaluate if the use of treaties is sufficient for self-organising this system, so as to achieve a level of stability when this game is played iteratively in an \textit{N-player} scenario.