\chapter{Conclusion}\label{conclusion}

This project has involved the design and implementation of six different agents with various strategies to self-organise and solve a common resource management problem, where the common resource, in this case, is the scarcity of food available to agents in the tower.

It has been shown that in order for self-organisation to be achieved within the tower, communication is necessary between all present agents through both messaging and the signing of treaties to simulate localised governance. Agents which were successful in achieving self-organisation did so by gaining knowledge of both the tower and other agents, and applying this knowledge appropriately. Without this, the most logical behaviour of an agent is to act purely selfishly, giving itself the best chance of survival in a random system.

The codebase for this project can be found at \url{https://github.com/SOMAS2021/SOMAS2021}.